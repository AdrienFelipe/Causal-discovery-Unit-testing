% Definimos el estilo del documento
\documentclass[12pt,a4paper,spanish]{book}
% Utilizamos el paquete para utilizar español
\usepackage[spanish]{babel}
% Utilizamos el paquete para gestionar acentos
\usepackage[latin1]{inputenc}
%Utilizamos el paquete para incorporar graficos postcript
\usepackage[dvips,final]{epsfig}
% Definimos la zona de la pagina ocupada por el texto
\oddsidemargin 0.1cm \headsep 0.5cm \textwidth=15.5cm
\textheight=22cm
%Empieza el documento
\begin{document}
% Definimos titulo, autor, fecha, generamos titulo e indice de
contenidos
\title{TITULO DE MI PROYECTO}
\author{MI NOMBRE}
\date{}
\maketitle
\tableofcontents
% Definimos una primera pagina para los agradecimientos
\newpage
\thispagestyle{empty}
\section*{Agradecimientos}
 Aqui ponemos los agradecimientos
% Empezamos capitulos
\chapter{Introducci\'on}
 La introducción es lo primero que se lee, pero habitualmente lo
último que se escribe. Pues su redacción
 depende de cómo se hayan escrito todas las otras secciones.
Normalmente la introducción incluye una
 descripción muy general del proyecto y termina con un desglose del
contenido de la memoria.
\chapter{Estado actual del tema}
 Descripción del estado actual del tema con referencia a trabajos
anteriores en el caso de proyectos que
 sean continuación o relacionados con otros proyectos.
\chapter{Metodología}
 Metodología a utilizar para el desarrollo del proyecto,
herramientas de análisis, etc..
\chapter{Recursos necesarios}
 Detallar los recursos hardware, software u otros necesarios para el
desarrollo del proyecto.
\chapter{Plan de trabajo y temporización}
Desarrollo del plan de trabajo desglosado en etapas, con una
estimación en cada etapa del tiempo de ejecución
\begin{itemize}
\item Etapa 1 de Mi Proyecto
\item Etapa 2 de Mi Proyecto
\item .................
\end{itemize}
A continuación vendrán las secciones donde se desarrollan cada una de
las etapas del proyecto
\chapter{Etapa 1 de Mi proyecto}
 Desarrollo de la etapa 1 de mi proyecto
\chapter{Etapa 2 de Mi proyecto}
 Desarrollo de la etapa 2 de mi proyecto
\chapter{Resultados y conclusiones}
 Desarrollo de los resultados y conclusiones del proyecto. Se debe
intentar resaltar el interés del proyecto
 y la calidad del trabajo realizado. Ha llegado el momento de
"vender" nuestro trabajo. Se deben incluir aspectos como:
\begin{itemize}
\item Calidad, dificultad y amplitud del trabajo desarrollado que
justifique el tiempo de dedicación al proyecto.
\item Aspectos integradores de las disciplinas de la titulación de
Ingeniero en Informática.
\item Impacto social. Utilidad del proyecto en el ámbito social
\item Facilidad de utilización de los resultados del proyecto por
terceras personas.
\item Publicidad de los resultados del proyecto a través de páginas
web, etc.. Cuando de los resultados del proyecto
 se derive un prototipo o programa de utilización se debe poner
a disposición del público en general una versión
 de demostración de dicho prototipo.
\item Cualquier otro mérito.
\end{itemize}
\chapter{Trabajo Futuro}
 Continuidad del trabajo realizado a través de una implementación,
o utilidad real del proyecto,
 o a través de otros proyectos fin de carrera.
% EMPIEZAN LOS APENDICES DEL PROYECTO
\appendix
\chapter{Manuales de usuario}
 En el caso de que el desarrollo (y/o naturaleza del proyecto haya
dado lugar a la creación de manuales
 de usuario, habrá que ponerlo aquí).
\chapter[Detalles Implementación]{Detalles técnicos sobre la
implementación del proyecto}
 En las secciones anteriores del proyecto, no se debe entrar en
demasiados detalles técnicos sobre
 cuestiones de implementación del proyecto pues dificultaría su
lectura y comprensión. Los detalles técnicos
 sobre la implementación del proyecto se incluyen
preferentemente en apéndices al final de la memoria.
\chapter[Redacción de Proyectos]{Comentarios Generales sobre la
redacción de los proyectos}
Por donde se empieza? Habitualmente se empieza a redactar en orden
inverso al que se lee el documento, primero
se puede empezar a rellenar la bibliografía utilizada que se irá
completando según avance el proyecto. A continuación,
a partir de un esqueleto inicial del proyecto, que puede estar
redactado, manuscrito, o simplemente en la cabeza del estudiante,
se empieza a redactar las partes más concretas del proyecto, que
tengamos más claras, y que sean lo más independiente posible
de la redacción de otras partes del proyecto. Por ejemplo, se puede
empezar por anexos donde se resuma las características
de una herramienta que utilizamos, etc.., a continuación empezamos a
redactar de manera individual los detalles de cada
una de las etapas en las que se constituye el proyecto, no tienen que
redactarse ordenadas según aparecen en el texto,
sobre la elección sobre cual empezar, siempre primará que sea una
parte que tengamos bien clara, y que hayamos delimitado
su contenido para que sea independiente de la redacción de las otras
etapas. Como verán, según vayan avanzando en la
redacción, cada vez verán las cosas más claras, y de forma natural
verán la forma de ir redactando las otras partes del
proyecto, hasta llegar a las secciones de introducción y conclusiones
y resultados que son las más delicadas de desarrollar,
pues son las más importantes y las que previsiblemente se van a leer
en mayor detalle las personas que lean el proyecto.
Una deficiente redacción de la introducción (que es donde se atrae al
lector sobre la importancia de lo que se va a hacer)
o una mala presentación de las conclusiones y resultados (que es
donde se transmite el mensaje de todo lo bueno que hay
en proyecto) pone en entredicho la calidad global del proyecto. Una
buena estrategia consiste en según se van redactando
las diferentes secciones del proyecto, ir haciendo un borrador de las
secciones de introducción y conclusiones y resultados,
poniendo las ideas sueltas, y en principio desordenadas, que nos
vayan surgiendo y que puedan ser de utilidad en la redacción
de estas secciones. Redactar bien tiene su dificultad y no todos los
días tenemos la inspiración adecuada, para esos días
negros, que no nos viene nada a la cabeza, lo mejor es dedicarse a
cosas más mecánicas que no requieren tanta concentración,
como puede ser completar la bibliografía, ir haciendo un manual de
usuario o un anexo técnico, etc..
Cuando empezamos a redactar, siempre es necesario tener en cuenta
algunos criterios básicos como son:
1. Escribir de cada cosa su esencia. Que es lo que es realmente
relevante en la sección que estoy redactando y esforzarme
en que ello quede claro
2. Ponerse en el lugar del potencial lector. El orden y la forma en
la redactamos no es sólo para que nosotros tengamos
claro lo que hacemos, es sobre todo para que una tercera persona que
lea el texto lo pueda tener, si cabe, más claro
que nosotros. Para ello hay que respetar un orden lógico en la forma
en que presentamos las cosas y no presuponer que
el lector conoce los entresijos de lo que estamos haciendo, hay que
evitar dar saltos en el vacío, por ejemplo dando
por supuesto conocimientos que el lector no tiene o alterando el
orden natural en que deben aparecer las cosas.
Algunas ideas sueltas sacadas del libro "Como elaborar y presentar un
trabajo escrito" cuyo autor es el profesor Santos Pérez:
El proyecto fin de carrera es un trabajo personal en el que el
estudiante debe demostrar que domina el tema, sabe organizarlo,
estructurarlo y elaborar en profundidad, y presentarlo en la forma
normalizada de un trabajo técnico o científico. Es la
ocasión que tiene el estudiante de demostrar que sabe analizar un
problema, sabe seleccionar la metodología y técnicas
apropiadas para reunir los datos, y alcanzar conclusiones razonables.
Un proyecto de esta naturaleza permite la evaluación
de la capacidad del estudiante para aplicar su conocimiento a un tema
concreto.
El proyecto fin de carrera debe ser
\begin{itemize}
\item Proyecto Personal: debe ser un producto de la reflexión,
investigación y esfuerzo del estudiante. Si se hace con
 reflexión, con investigación y esfuerzo personal, el
rendimiento que obtiene el estudiante es muy productivo y
 beneficioso para él y de una duración permanente.
\item Es un trabajo documentado: es decir, serio, científico, hay
que sustentar las afirmaciones con datos comprobables
 y lógicamente fundados.
\item Planificado: La elaboración de un proyecto fin de carrera es
un proceso complicado. Un trabajo de esta naturaleza
 requiere una planificación cuidadosa del tiempo: tiempo para
investigar y documentarse, tiempo para reflexionar,
 tiempo para corregir posibles desviaciones y finalmente tiempo
para redactarlo y presentarlo de forma adecuada.
\end{itemize}
El esquema final de un proyecto fin de carrera debe llenar las
siguientes características:
\begin{itemize}
\item Claridad: la claridad se consigue sobre todo con una nítida
división y distribución del esquema. Y a su vez esta
 claridad deriva también de la compresión en profundidad del
material recogido.
\item Convergencia hacia el objetivo: El secreto de la claridad
está en saber ordenar las partes del trabajo hacia el
 objetivo buscado; es decir, en lograr que cada punto del
esquema nos vaya encaminando con naturalidad hacia la meta
 enunciada en el título del trabajo.
\item Coherencia: las distintas partes, puntos o párrafos deben
estar trabados entre sí, concatenados, de forma que se 
 vayan preparando y completandose recíprocamente para
conseguir el efectode que cada punto sea consecuencia del otro,
 formando un todo orgánico y no una mera yuxtaposición de
partes; por el contrario que se vaya mostrando la conexión
 y la coherencia lógica de los distintos aspectos tratados
\item Conformidad con el objetivo: La estructura del esquema final
debe resaltar lo más importante y debe dejar en la
 penumbra los accesorios.
\item Elegancia: en la distribución del esquema, debe guardarse una
cierta simetría y proporción. La elegancia no debe
 subordinarse a la claridad y a la verdad; pero hay una
elegancia no sólo formal, sino de concepción y elegancia
 que contribuye significativamente a conseguir la armonía y
transparencia en la transmisión del contenido principal
 del tema. Conviene resaltar esta elegancia sobre todo ahora
que nos encontramos en un mundo de zafiedad y donde se
 hace gala del caos mental como norma de actuación.
\item El descanso inteligente: Una vez que se ha acabado la primera
redacción del proyecto se sugiere tomarse unos
 días de descanso suficientes para que la cabeza descanse del
tema. Con este descanso se adquiere perspectiva,
 y aumenta la objetividad y sentido crítico del autor.
\end{itemize}
% Aqui va la Bibliografía utilizada por el proyecto.
\begin{thebibliography}{1}
\bibitem{La86} Leslie Lamport {\em LaTex : A document Preparation
System}. Addison-Wesley, 1986.
\bibitem{Ro93} Christian Rolland {\em LaTex guide pratique}. AddisonWesley, 1993.
\end{thebibliography}
% Termina el documento
\end{document}